\documentclass[dvipdfmx]{ujarticle}
\usepackage{eee}

\begin{document}
\title{令和2年 電磁気学II 第1回小テスト}
\date{}
\author{大山主朗}

\maketitle

\section{以下の(a)及び(d)に示す物理定数は電磁気学を修めた者であれば常識的に覚えていなければならない数値である.それぞれの値を示せ.}
\begin{enumerate}[(a)]
	\item 真空の誘電率$\varepsilon_{0}:8.854 \times 10^{-12}\,\rm{F/m}$
	\item 真空の透磁率$\mu_{0}:1.257 \times 10^{-6}\,\rm{H/m}$
	\item 電子の電荷$e:-1.602 \times 10^{-19}\,\rm{C}$
	\item 電子の静止質量$m:9.109\times 10^{-31}\,\rm{kg}$
\end{enumerate}

\section{$AB=BC=a$, $\angle B=90^{\circ}$の直角二等辺三角形 ABC がある.いま各頂点に点磁荷$m$が存在するとき,以下の各問いに答えよ.}
\begin{enumerate}[(a)]
	\item 頂点Bに存在する点磁荷にはたらく力$F_{B}$を求めよ.
	\begin{align*}
		\boldsymbol{F}_{BA}&=\frac{1}{4\pi \mu_{0}}\frac{m^{2}}{\left((-a)^{2} \right)^{3/2}}\left(-a\boldsymbol{j}\right)\\
	&=-\frac{m^{2}}{4\pi \mu_{0}a^{2}}\boldsymbol{j}\,[\rm{N}]\\
	\boldsymbol{F}_{BC}&=\frac{1}{4\pi \mu_{0}}\frac{m^{2}}{\left(a^{2} \right)^{3/2}}\left(-a\boldsymbol{i}\right)\\
	&=-\frac{m^{2}}{4\pi \mu_{0}a^{2}}\boldsymbol{i}\,[\rm{N}]\\
	\boldsymbol{F}_{B}&=\boldsymbol{F}_{BA}+\boldsymbol{F}_{BC}\\
	&=-\frac{m^{2}}{4\pi \mu_{0}a^{2}}\left(\boldsymbol{i}+\boldsymbol{j}\right)\,[\rm{A/m}]\\
	|\boldsymbol{F}_{B}|&=\frac{m^{2}}{4\pi \mu_{0}a^{2}}\sqrt{2}\,[\rm{N}]\quad 頂点Bから辺ABの反対側向き
	\end{align*}
	\item 頂点Aに存在する点磁荷にはたらく力$F_{A}$を求めよ.
	\begin{align*}
	\boldsymbol{F}_{AB}&=\frac{1}{4\pi \mu_{0}}\frac{m^{2}}{\left(a^{2} \right)^{3/2}}\left(a\boldsymbol{j}\right)\\
	&=\frac{m^{2}}{4\pi \mu_{0}a^{2}}\boldsymbol{j}\,[\rm{A/m}]\\
	\boldsymbol{F}_{AC}&=\frac{1}{4\pi \mu_{0}}\frac{m^{2}}{\left((-a)^{2}+a^{2} \right)^{3/2}}\left(-a\boldsymbol{i}+a\boldsymbol{j}\right)\\
	&=\frac{m^{2}}{8\sqrt{2} \pi \mu_{0} a^{2}}\left( -\boldsymbol{i}+\boldsymbol{j}\right)\,[\rm{A/m}]\\
	\boldsymbol{F}_{A}&=\boldsymbol{F}_{AB}+\boldsymbol{F}_{AC}\\
	&=\frac{m^{2}}{4\pi \mu_{0}a^{2}}\left\{-\frac{1}{2\sqrt{2}}\boldsymbol{i}+\left(1+\frac{1}{2\sqrt{2}}\right)\boldsymbol{j}\right\}\\
	&=\frac{m^{2}}{16\pi \mu_{0}a^{2}}\left\{-\sqrt{2}\boldsymbol{i}+\left(4+\sqrt{2}\right)\boldsymbol{j} \right\}\,[\rm{N}]\\
	|\boldsymbol{F}_{A}|&=\frac{m^{2}}{16\pi \mu_{0}a^{2}}\sqrt{2+16+8\sqrt{2}+2}\\
	&=\frac{m^{2}}{16\pi \mu_{0}a^{2}}\sqrt{20+8\sqrt{2}}\\
	&=\frac{m^{2}}{8\pi \mu_{0}a^{2}}\sqrt{5+2\sqrt{2}}\,[\rm{A/m}]\quad 左斜め上向き
	\end{align*}
	\item 直角三角形ABCの内接円の半径$r$を求めよ.
	\begin{align*}
		r&=\frac{2S}{|AB|+|BC|+|AC|}\\
		&=\frac{2\cdot \frac{1}{2}a^{2}}{a+a+a\sqrt{2}}
	\end{align*}
	\item 頂点Aに存在する点磁荷が直角三角形ABCの内心につくる磁界$H_{A}$を求めよ.
	\begin{align*}
	\end{align*}
\end{enumerate}

\section{$xy$直交座標系において,同量異符号の点磁荷$\pm m$が距離$l$に固定された磁気双極子が存在する.このとき以下の問いに答えよ.}
\begin{enumerate}[(a)]
	\item 点Aに存在する磁荷$-m$が点P$(x_0,y_0)$に作る磁界$H_{1}$を求めよ.また,$H_{1}$を$x$方向成分$H_{x1}$と$y$方向成分$H_{y1}$に分解せよ.
	\begin{align*}
		%\boldsymbol{H}_{1}&=\frac{1}{4\pi \mu_{0}}\frac{-m}{\left(\left(x_{0}+\frac{l}{2}\right)^{2}+y_{0}^{2}\right)^{3/2}} \left\{ \left(x_{0}+\frac{l}{2}\right)\boldsymbol{i}+y_{0}\boldsymbol{j}\right\}\,[\rm{A/m}]
	\end{align*}
	\item 点Bに存在する磁荷$+m$が点P$(x_0,y_0)$に作る磁界$H_{2}$を求めよ.また,$H_{2}$を$x$方向成分$H_{x2}$と$y$方向成分$H_{y2}$に分解せよ.
	\begin{align*}
		%\boldsymbol{H}_{2}&=\frac{1}{4\pi \mu_{0}}\frac{m}{\left(\left(x_{0}-\frac{l}{2}\right)^{2}+y_{0}^{2}\right)^{3/2}} \left\{ \left(x_{0}-\frac{l}{2}\right)\boldsymbol{i}+y_{0}\boldsymbol{j}\right\}\,[\rm{A/m}]
	\end{align*}
	\item 点Pでの磁界$H$の$x$方向成分$H_{x}$と$y$方向成分$H_{y}$をそれぞれ求めよ.
	\begin{align*}
		%\boldsymbol{H}&=\boldsymbol{H}_{1}+\boldsymbol{H}_{2}\\
		%&=\frac{m}{4\pi \mu_{0}}\left[\frac{-1}{\left(\left(x_{0}+\frac{l}{2}\right)^{2}+y_{0}^{2}\right)^{3/2}} \left\{ \left(x_{0}+\frac{l}{2}\right)\boldsymbol{i}+y_{0}\boldsymbol{j}\right\}+\frac{1}{\left(\left(x_{0}-\frac{l}{2}\right)^{2}+y_{0}^{2}\right)^{3/2}} \left\{ \left(x_{0}-\frac{l}{2}\right)\boldsymbol{i}+y_{0}\boldsymbol{j}\right\}\right]\,[\rm{A/m}]
	\end{align*}
	\item 磁気双極子モーメント$\boldsymbol{M}$の大きさと方向を求めよ.
	\begin{align*}
		%\boldsymbol{M}&=m\boldsymbol{l}\\
		%&=ml\boldsymbol{i}\,[\rm{Wb\cdot m}]
	\end{align*}
	\item 点Pが原点Oより十分遠方にあると仮定すると,$\sqrt{(x_{0}-l/2)^{2}+y_{0}^{2}}\simeq \sqrt{x_{0}^{2}+y_{0}^{2}}$及び$\sqrt{(x_{0}+l/2)^{2}+y_{0}^{2}} \simeq \sqrt{x_{0}^{2}+y_{0}^{2}}$と近似できる.このことを用いて(c)にて得た磁界$H_{x}$及び$H_{y}$を簡略化せよ.
	\begin{align*}
		%\boldsymbol{H}&\simeq -\frac{1}{4\pi \mu_{0}}\frac{m}{\left(x_{0}^{2}+y_{0}^{2}\right)^{3/2}} \boldsymbol{i}\,[\rm{A/m}]
	\end{align*}
	\item $y$方向に一様な磁界$\boldsymbol{H}_{0}$が存在するとき,磁気双極子にはたらくトルク$T$を求めよ.
	\begin{align*}
		\boldsymbol{T}&=\boldsymbol{M}H_{0}\sin \theta \\
		&=ml\boldsymbol{i}\sin \frac{\pi}{2}\\
		&=ml\boldsymbol{i}\\
		|\boldsymbol{T}|&=ml\,[\rm{Wb\cdot m}]
	\end{align*}
\end{enumerate}

\section{磁化されていない強磁性体に磁界$H$を外部から印加し,強磁性体内部での磁束密度$B$を観測すると,図3に示すような結果が得られた.このとき,図中の行程1:点$\rm{O}\to$点$\rm{P_{1}}$,行程 2:点$\rm{P_{1}}\to$点$\rm{P_{2}}$,行程3:点$\rm{P_{2}}\to$点$\rm{P_{3}}$,行程4:点$\rm{P_{3}}\to$点P4,行程5:点P4 $\to$点$\rm{P_{5}}$, 行程6:点$\rm{P_{5}}\to$点P6,行程7:点$\rm{P_{6}}\to$点P1の7つの行程に着目して,測定結果を説明せよ.}

\section{強磁性体,弱磁性体,常磁性体,反磁性体の4つの磁性体の性質を,「比透磁率$\mu_{s}$」と「磁化率$\chi$」という2つの語句を両方用いて説明せよ.}
\end{document}