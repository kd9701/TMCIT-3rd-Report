\documentclass[dvipdfmx]{ujarticle}
\usepackage{eee}

\begin{document}
\title{令和3年 電磁気学II}
\date{}
\author{大山主朗}

\maketitle

\section*{令和3年 電磁気学II 前期中間試験}
\section{以下の(a)及び(d)に示す物理定数は電磁気学を修めた者であれば常識的に覚えていなければならない数値である.それぞれの値を示せ.}
\begin{enumerate}[(a)]
	\item 真空の誘電率$\varepsilon_{0}:8.854 \times 10^{-12}\,\rm{F/m}$
	\item 真空の透磁率$\mu_{0}:1.257 \times 10^{-6}\,\rm{H/m}$
	\item 電子の電荷$e:-1.602 \times 10^{-19}\,\rm{C}$
	\item 電子の静止質量$m:9.109\times 10^{-31}\,\rm{kg}$
\end{enumerate}

\section{$xy$直交座標系において,同量異符号の点磁荷$\pm m$が距離$l$に固定された磁気双極子が存在する.このとき以下の問いに答えよ.ただし,$x$ 方向の基準ベクトルを$\boldsymbol{i}$,$y$方向の基準ベクトルを$\boldsymbol{j}$とする}
\begin{enumerate}[(a)]
	\item 点Aに存在する磁荷$-m$が点P$(x_0,y_0)$に作る磁界$\boldsymbol{H}_{1}$を求めよ.
	\begin{align*}
		\boldsymbol{H}_{1}&=\frac{1}{4\pi \mu_{0}}\frac{-m}{\left(\left(x+a\right)^{2}+y^{2}\right)^{3/2}} \left\{ \left(x+a\right)\boldsymbol{i}+y\boldsymbol{j}\right\}\,[\rm{A/m}]
	\end{align*}
	\item 点Bに存在する磁荷$+m$が点P$(x_0,y_0)$に作る磁界$\boldsymbol{H}_{2}$を求めよ.
	\begin{align*}
		\boldsymbol{H}_{2}&=\frac{1}{4\pi \mu_{0}}\frac{m}{\left(\left(x-a\right)^{2}+y^{2}\right)^{3/2}} \left\{ \left(x-a\right)\boldsymbol{i}+y\boldsymbol{j}\right\}\,[\rm{A/m}]
	\end{align*}
	\item 点Pでの磁界$\boldsymbol{H}$を求めよ.
	\begin{align*}
		\boldsymbol{H}&=\boldsymbol{H}_{1}+\boldsymbol{H}_{2}\\
		&=\frac{m}{4\pi \mu_{0}}\left[\frac{-1}{\left(\left(x+a\right)^{2}+y^{2}\right)^{3/2}} \left\{ \left(x+a\right)\boldsymbol{i}+y\boldsymbol{j}\right\}+\frac{1}{\left(\left(x-a\right)^{2}+y^{2}\right)^{3/2}} \left\{ \left(x-a\right)\boldsymbol{i}+y\boldsymbol{j}\right\}\right]\,[\rm{A/m}]
	\end{align*}
	\item 磁気双極子モーメント$\boldsymbol{M}$を求めよ.
	\begin{align*}
		\boldsymbol{M}&=m\boldsymbol{l}\\
		&=ml\boldsymbol{i}\,[\rm{Wb\cdot m}]
	\end{align*}
	\item 点Pが原点Oより十分遠方にあると仮定すると,$\sqrt{(x-a)^{2}+y^{2}}\simeq \sqrt{x^{2}+y^{2}}$及び$\sqrt{(x+a)^{2}+y^{2}}\simeq \sqrt{x^{2}+y^{2}}$と近似できる.このことを用いて(c)にて得た磁界$\boldsymbol{H}$を簡略化せよ.
	\begin{align*}
		\boldsymbol{H}&\simeq -\frac{1}{4\pi \mu_{0}}\frac{ml}{\left(x^{2}+y^{2}\right)^{3/2}} \boldsymbol{i}\,[\rm{A/m}]\\
		&\left(\simeq -\frac{\boldsymbol{M}}{4\pi \mu_{0}r^{3}}\,[\rm{A/m}]\right)
	\end{align*}
	\item $y$方向に一様な磁界$\boldsymbol{H}_{0}$が存在するとき,磁気双極子にはたらくトルク$T$を求めよ.
	\begin{align*}
		\boldsymbol{T}&=\boldsymbol{M}H_{0}\sin \theta \\
		&=ml\boldsymbol{i}\sin \frac{\pi}{2}\\
		&=ml\boldsymbol{i}\\
		|\boldsymbol{T}|&=ml\,[\rm{Wb\cdot m}],x軸右方向
	\end{align*}
\end{enumerate}

\section{$xyz$直交座標空間において,$xy$平面内に原点Oを中心とする半径$a$の円形ループ電流$I$が流れており,$z$軸上に点P$(0, 0, h)$がある.このとき,点Pに発生する磁界$\boldsymbol{H}$を求めよ.}
	\begin{align*}
	\end{align*}

\section{$xyz$直角座標空間において,$y$軸上の点$\rm{A}(0, c_{1}, 0)$から点$\rm{B}(0, c_{2}, 0)$まで$y$軸に沿って直線状に流れる電流$I$がある.このとき,$x$軸上の点$\rm{P}(a, 0, 0)$に発生する磁界$\boldsymbol{H}$を求めよ.また,電流$I$の始点 Aと終点Bの座標がそれぞれ$(0, -\infty, 0), (0, \infty, 0)$となった場合の点$\rm{P}$に発生する磁界$\boldsymbol{H}$を求めよ.}
	\begin{align*}
	\end{align*}

\section{半径$a$の半円と半径$b$の半円が接続された導体に電流$I$が流れている.このとき,半円の中心Oに発生する磁界$\boldsymbol{H}$を求めよ.}
	\begin{align*}
	\end{align*}

\section{$xyz$直交座標系においてTVアニメ版だと「第 5 使徒ラミエル」,新劇場版だと「第 6 の使徒」と呼ばれるような$(\pm a, 0, 0), (0, \pm a , 0), (0, 0, \pm a)$の点を通る正八面体がある.この正八面体の各辺に図のように電流$I$が流れている.このとき,原点Oに発生する磁界$\boldsymbol{H}$を求めよ.}
	\begin{align*}
	\end{align*}

\section{磁化されていない強磁性体に磁界$H$を外部から印加し,強磁性体内部での磁束密度$B$を観測すると,図3に示すような結果が得られた.このとき,図中の行程1:点$\rm{O}\to$点$\rm{P_{1}}$,行程 2:点$\rm{P_{1}}\to$点$\rm{P_{2}}$,行程3:点$\rm{P_{2}}\to$点$\rm{P_{3}}$,行程4:点$\rm{P_{3}}\to$点P4,行程5:点P4 $\to$点$\rm{P_{5}}$, 行程6:点$\rm{P_{5}}\to$点P6,行程7:点$\rm{P_{6}}\to$点P1の7つの行程に着目して,測定結果を説明せよ.}

\end{document}
