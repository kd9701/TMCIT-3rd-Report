\documentclass[dvipdfmx]{ujarticle}
\usepackage{eee}

\begin{document}
\title{令和3年 電磁気学II}
\date{}
\author{大山主朗}

\maketitle

\section*{令和3年 電磁気学II 前期中間試験}
\section{以下の(a)及び(d)に示す物理定数は電磁気学を修めた者であれば常識的に覚えていなければならない数値である.それぞれの値を示せ.}
\begin{enumerate}[(a)]
	\item 真空の誘電率$\varepsilon_{0}:8.854 \times 10^{-12}\,\rm{F/m}$
	\item 真空の透磁率$\mu_{0}:1.257 \times 10^{-6}\,\rm{H/m}$
	\item 電子の電荷$e:-1.602 \times 10^{-19}\,\rm{C}$
	\item 電子の静止質量$m:9.109\times 10^{-31}\,\rm{kg}$
\end{enumerate}

\section{$xy$直交座標系において,同量異符号の点磁荷$\pm m$が距離$l$に固定された磁気双極子が存在する.このとき以下の問いに答えよ.}
\begin{enumerate}[(a)]
	\item 点Aに存在する磁荷$-m$が点P$(x_0,y_0)$に作る磁界$H_{1}$を求めよ.また,$H_{1}$を$x$方向成分$H_{x1}$と$y$方向成分$H_{y1}$に分解せよ.
	\begin{align*}
		\boldsymbol{H}_{1}&=\frac{1}{4\pi \mu_{0}}\frac{-m}{\left(\left(x_{0}+\frac{l}{2}\right)^{2}+y_{0}^{2}\right)^{3/2}} \left\{ \left(x_{0}+\frac{l}{2}\right)\boldsymbol{i}+y_{0}\boldsymbol{j}\right\}\,[\rm{A/m}]\\
		|\boldsymbol{H}_{1}|&=\frac{1}{4\pi \mu_{0}}\frac{m}{\left(x_{0}+\frac{l}{2}\right)^{2}+y_{0}^{2}}\,[\rm{A/m}] \\
		|\boldsymbol{H}_{x1}|&=\frac{1}{4\pi \mu_{0}}\frac{m}{\left(\left(x_{0}+\frac{l}{2}\right)^{2}+y_{0}^{2}\right)^{3/2}}\left(x_{0}+\frac{l}{2}\right) \,[\rm{A/m}] \quad x軸上方向\\
		|\boldsymbol{H}_{y1}|&=\frac{1}{4\pi \mu_{0}}\frac{m}{\left(\left(x_{0}+\frac{l}{2}\right)^{2}+y_{0}^{2}\right)^{3/2}}\,y_{0}\,[\rm{A/m}] \quad y軸下方向
	\end{align*}
	\item 点Bに存在する磁荷$+m$が点P$(x_0,y_0)$に作る磁界$H_{2}$を求めよ.また,$H_{2}$を$x$方向成分$H_{x2}$と$y$方向成分$H_{y2}$に分解せよ.
	\begin{align*}
		\boldsymbol{H}_{2}&=\frac{1}{4\pi \mu_{0}}\frac{m}{\left(\left(x_{0}-\frac{l}{2}\right)^{2}+y_{0}^{2}\right)^{3/2}} \left\{ \left(x_{0}-\frac{l}{2}\right)\boldsymbol{i}+y_{0}\boldsymbol{j}\right\}\,[\rm{A/m}]\\
		|\boldsymbol{H}_{2}|&=\frac{1}{4\pi \mu_{0}}\frac{m}{\left(x_{0}-\frac{l}{2}\right)^{2}+y_{0}^{2}}\,[\rm{A/m}] \\
		|\boldsymbol{H}_{x2}|&=\frac{1}{4\pi \mu_{0}}\frac{m}{\left(\left(x_{0}-\frac{l}{2}\right)^{2}+y_{0}^{2}\right)^{3/2}}\left(x_{0}-\frac{l}{2}\right) \,[\rm{A/m}] \quad x軸正方向\\
		|\boldsymbol{H}_{y2}|&=\frac{1}{4\pi \mu_{0}}\frac{m}{\left(\left(x_{0}-\frac{l}{2}\right)^{2}+y_{0}^{2}\right)^{3/2}}\,y_{0}\,[\rm{A/m}] \quad y軸正方向
	\end{align*}
	\item 点Pでの磁界$H$の$x$方向成分$H_{x}$と$y$方向成分$H_{y}$をそれぞれ求めよ.
	\begin{align*}
		|\boldsymbol{H}_{x}|&=|\boldsymbol{H}_{x1}|+|\boldsymbol{H}_{x2}|\\
		&=\frac{m}{4\pi \mu_{0}}\left\{\frac{1}{\left(\left(x_{0}-\frac{l}{2}\right)^{2}+y_{0}^{2}\right)^{3/2}}\left(x_{0}-\frac{l}{2}\right)-\frac{1}{\left(\left(x_{0}+\frac{l}{2}\right)^{2}+y_{0}^{2}\right)^{3/2}}\left(x_{0}+\frac{l}{2}\right)\right\}\,[\rm{A/m}]\\
		|\boldsymbol{H}_{y}|&=|\boldsymbol{H}_{y1}|+|\boldsymbol{H}_{y2}|\\
		&=\frac{m}{4\pi \mu_{0}}y_{0}\left\{\frac{1}{\left(\left(x_{0}-\frac{l}{2}\right)^{2}+y_{0}^{2}\right)^{3/2}}-\frac{1}{\left(\left(x_{0}+\frac{l}{2}\right)^{2}+y_{0}^{2}\right)^{3/2}}\right\}\,[\rm{A/m}]
	\end{align*}
	\item 磁気双極子モーメント$\boldsymbol{M}$の大きさと方向を求めよ.
	\begin{align*}
		\boldsymbol{M}&=m\boldsymbol{l}\\
		&=ml\boldsymbol{i}\,[\rm{Wb\cdot m}]\\
		|\boldsymbol{M}|&=ml\,[\rm{Wb\cdot m}]\quad x軸正方向
	\end{align*}
	\item 点Pが原点Oより十分遠方にあると仮定すると,$\sqrt{(x_{0}-l/2)^{2}+y_{0}^{2}}\simeq \sqrt{x_{0}^{2}+y_{0}^{2}}$及び$\sqrt{(x_{0}+l/2)^{2}+y_{0}^{2}} \simeq \sqrt{x_{0}^{2}+y_{0}^{2}}$と近似できる.このことを用いて(c)にて得た磁界$H_{x}$及び$H_{y}$を簡略化せよ.
	\begin{align*}
		|\boldsymbol{H}_{x}|&\simeq \frac{ml}{4\pi \mu_{0}\left(x_{0}^{2}+y_{0}^{2}\right)^{3/2}}\,[\rm{A/m}] \quad x軸左方向\\
		|\boldsymbol{H}_{y}|&\simeq 0\,[\rm{A/m}]
	\end{align*}
	\item $y$方向に一様な磁界$\boldsymbol{H}_{0}$が存在するとき,磁気双極子にはたらくトルク$T$を求めよ.
	\begin{align*}
		\boldsymbol{T}&=\boldsymbol{M}H_{0}\sin \theta \\
		&=ml\boldsymbol{i}H_0\sin \frac{\pi}{2}\\
		&=mlH_{0}\boldsymbol{i}\\
		|\boldsymbol{T}|&=mlH_{0}\,[\rm{Wb\cdot m}]
	\end{align*}
\end{enumerate}

\section{磁化されていない強磁性体に磁界$H$を外部から印加し,強磁性体内部での磁束密度$B$を観測すると,図3に示すような結果が得られた.このとき,図中の行程1:点$\rm{O}\to$点$\rm{P_{1}}$,行程 2:点$\rm{P_{1}}\to$点$\rm{P_{2}}$,行程3:点$\rm{P_{2}}\to$点$\rm{P_{3}}$,行程4:点$\rm{P_{3}}\to$点P4,行程5:点P4 $\to$点$\rm{P_{5}}$, 行程6:点$\rm{P_{5}}\to$点P6,行程7:点$\rm{P_{6}}\to$点P1の7つの行程に着目して,測定結果を説明せよ.}

\section{強磁性体,弱磁性体,常磁性体,反磁性体の4つの磁性体の性質を,「比透磁率$\mu_{s}$」と「磁化率$\chi$」という2つの語句を両方用いて説明せよ.}
強磁性体は磁化率$\chi$が0よりかなり大きく,透磁率$\mu_{s}$が1よりかなり大きい磁化されやすい磁性体を指す.そのため,印加した磁界と同じ方向に磁化され,その大きさも大きい.

弱磁性体は磁化率$\chi$が0より大きく,透磁率$\mu_{s}$より小さい磁性体である.

常磁性体は磁化率$\chi$が0より大きく,透磁率$\mu_{s}$は1未満の磁性体を指す.そのため,印加した磁界と同じ方向に磁化され,その大きさは大きくない.

反磁性体は磁化率$\chi$が0より小さく,透磁率$\mu_{s}$が1より小さい磁性体を指す.そのため,印加した磁界と逆方向に磁化され,その大きさは小さい.

\clearpage
\setcounter{section}{0}
\section*{令和3年 電磁気学II 第2回小テスト}

\end{document}
