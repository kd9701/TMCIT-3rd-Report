\documentclass[11pt,dvipdfmx]{jarticle}

\usepackage{eee}
\usepackage{subfig}

\renewcommand{\labelenumi}{\alph{enumi}}
\renewcommand{\labelenumii}{\roman{enumii}}

\begin{document}
% トップページを書く
\begin{jikkenTitle}
 \gakunen{3} % 学年を記述。この行で全体の枠を表示
 \numTitle{3}{電力と力率} % 実験番号、タイトルを記述
 \subTitle{} % サブタイトルがあれば記述
 \jikkenbi{平成  年  月  日(  )} % 実験日を記述
 \jikkenbiII{平成  年  月  日(  )} % 実験日を記述(二日目がある場合。ない場合はこの行をコメントアウト)
 \kyoudou{共同実験者名} % 共同実験者名を記述
 \kyoudouII{} % その他の共同実験者名を記述
 \yoteibi{/  }% 予定日を記述
 \yoteibiII{}% 予定日2を記述
 \yoteibiIII{}% 予定日3を記述
 \hanNumberName{}{}{} % 班番号・学生番号・氏名を記述。この行でタイトルページの描画を終了
\end{jikkenTitle}

\section{目的}
本実験では
\begin{itemize}
	\item 単相交流回路における電圧・電流・電力・力率を測定するための結線方法を学ぶ。
	\item 単相電力計と力率計の扱い方を習得する。
	\item 有効電力と力率、皮相電力と無効電力に関する理解を深める
\end{itemize}
ことを目的とする。

\section{原理}
\subsection{瞬時電力}
インピーダンス$\dot{Z}\,[\Omega]$へ印加された時刻$t\,[\mathrm{s}]$における交流電圧$v(t)\,[\mathrm{V}]$と、$\dot{Z}$に流れる交流電流$i(t)\,[\mathrm{A}]$がそれぞれ次式で表されるとする。
\begin{eqnarray}
	v(t) &=& V_m\sin(\omega t+\theta_V)\\
	i(t) &=& I_m\sin(\omega t + \theta_I)
\end{eqnarray}
ここで、$V_m$、$I_m$は最大値、$\omega\,[\mathrm{rad/s}]$は角周波数、$\theta_V\,[\mathrm{rad}]$と$\theta_I\,[\mathrm{rad}]$はそれぞれの位相である。
この$v(t)$と$i(t)$の積を瞬時電力$p(t)$と呼び、次式で表される。
\begin{eqnarray}
	p(t)&=& v(t)i(t)\nonumber\\
	&=&V_mI_m\sin(\omega t+\theta_V)\sin(\omega t + \theta_I)\nonumber\\
	&=& \frac{V_mI_m}{2}\Bigl(\cos(2\omega t + \theta_I+\theta_V)+\cos(\theta_I - \theta_V)\Bigr)
	\label{eq:ip}
\end{eqnarray}


\subsection{有効電力と力率}
\weq{ip}は$v(t)$や$i(t)$の2倍の角速度を持つ周期関数であることが確認できる。
そのため、時間的な平均を算出することができ、この値を有効電力$P\,[\mathrm{W}]$と呼ぶ。
\begin{eqnarray}
	P &=& \frac{1}{T}\int_0^T \frac{V_mI_m}{2}\Bigl(\cos(2\omega t + \theta_I+\theta_V)+\cos(\theta_I - \theta_V)\Bigr) dt\nonumber\\
	&=&\frac{V_mI_m}{2}\cos(\theta_I - \theta_V)
\end{eqnarray}

この上式が得られたとき、交流回路における実効値表現に置き換えると
\begin{equation}
	P = VI\cos\theta
	\label{eq:power}
\end{equation}
を得ることができる。ここで、$V$、$I$はそれぞれの実効値、$\theta=\theta_I-\theta_V$である。
\weq{power}の右辺は電圧と電流の実効値の積と、$\cos\theta$から構成されている。$\theta$は$\dot{Z}$の実部(抵抗)と虚部(リアクタンス)の比によって決定される値であり、
\begin{equation}
	-\frac{\pi}{2}\leq\theta\leq\frac{\pi}{2}
\end{equation}
であるので、
\begin{equation}
	0\leq\cos\theta\leq 1
\end{equation}
の不等式が成立する。

以上の関係から、インピーダンス$\dot{Z}$の端子電圧と流れる電流値の積とは必ずしも等しくなく、有効に消費される電力の比が$\cos\theta$に相当することが分かる。
この比として見なせる$\cos\theta$を力率、$\theta$を力率角と呼ぶ。

\subsection{無効電力と皮相電力}
\weq{ip}において、インピーダンスがリアクタンス成分のみ($\dot{Z}=jX$)の場合について考える。
この時、電圧と電流の位相差$\theta_I-\theta_V$は$\pm\pi/2$となり、括弧内の第二項の値は0となる。
従って、瞬時電力$p(t)$の振る舞いは平均値が0の正弦波(あるいは余弦波)になることが分かる。
これは、電源から負荷へ、負荷から電源へ電力供給が交互に行われていることを示し、電力として消費されず仕事をしない。
この電力を無効電力$Q$とよび、単位には$\mathrm{var}$(バール)を用い、次式で計算される。
\begin{equation}
	Q = VI \sin \theta
\end{equation}

電圧の実効値と電流の実効値の積$VI$は、インピーダンス$\dot{Z}$が純抵抗(リアクタンス$X=0$)の場合にのみ有効電力と等しくなり、それ以外の場合では$VI> P$となる。
この、見かけ上の電力を皮相電力$S$とよび、単位には$\mathrm{VA}$(ボルトアンペア)を用いる。
また、皮相電力と有効電力、無効電力には次の関係が成り立つ。
\begin{eqnarray}
	S &=& VI\nonumber\\
	&=& \sqrt{P^2+Q^2}
\end{eqnarray}


\section{方法}
\subsection{使用器具}
今回の実験で使用した器具を

\subsection{実験手順}


\section{結果}
\section{考察}
\section{結論}

\begin{thebibliography}{9}%参考文献数が10以上の場合は9を99に変更
	%\bibitem{xxx}の引用を本文中で行うには\cite{xxx}と記述。
	\bibitem{a} 著者名, 書名,出版社,発行年.
\end{thebibliography}


\end{document}